\part{Compte rendu du stage}

\section{Développement du script de base pour la sauvegarde et le chiffrement des données}

\subsection{Convention de l'ANSSI}

L'Agence Nationale de la Sécurité des Systèmes d'Information recommande l'utilisation de
l'algorithme de hachage \textbf{SHA-256} et l'usage de clé \textbf{RSA 2048 bits}.

\subsection{Consigne}

Le script de base devra être développé en \textit{Python 2.7} et devra remplir les conditions
suivantes :

\begin{enumerate}
     \item Indentation de 4 espaces, pas de tabulations.
     \item Deux répertoires : \textbf{src} et \textbf{dest}, le fichiers ne doivent pas être
     copiés de \textit{src} vers \textit{dest}, ce dernier répertoire ne devra contenir que du
     contenu chiffré.
     \item Calculer la somme SHA-256 des fichiers sources et l'afficher.
     \item Chiffrer les fichiers avec \textit{GnuPG} dans \textit{dest}.

     \begin{enumerate}
          \item Si le fichier n'existe pas, le créer.
          \item Si le fichier existe le nommer selon ce pattern : \textit{filename.X.gpg} où X
          varie de 1 à 5 (à la 6è itération, on supprime le fichier).
     \end{enumerate}

     \item Placer le code source dans un dépôt \textit{git}.
     \item Utiliser \textit{unittest} pour les tests unitaires.
\end{enumerate}

\subsection{Conclusion}

Cette partie du logiciel a été développée sous la forme d'un paquet \textit{Python}.

A l'aide d'un unique objet, on est capable de récupérer la liste des fichiers du répertoire
source, de calculer leurs sommes SHA-256 et de les chiffrer via \textit{GnuPG} :

\begin{verbatim}
from src.filemanager import FileManager

fm = FileManager.FileManager (srcpath, destpath)
filelist = fm.read_entries ()
checksum = fm.hash_entries (filelist)
fm.gpg_encrypt (filelist) # encrypt
\end{verbatim}

\section{Gestion de la base de données SQL}

La base de données SQL sera utilisée pour indexer les fichiers chiffrés sur l'espace de stockage.
Grâce à elle, on ne chiffrera pas plusieurs fois un même fichier, et on ne transférera pas sur
l'hôte distant plusieurs fois les mêmes données.

Il est donc impératif que la base de données contienne les informations suivantes :

\begin{description}
     \item[hash:] Le hash SHA-256 du fichier non chiffré, il sera utilisé en tant qu'index de la
               base de données SQL.
     \item[path:] Le chemin d'accès vers le fichier chiffré, ainsi pour un fichier chiffré on a le
               hash du fichier non chiffré qui y est associé.
     \item[sent:] La date à laquelle le fichier a été envoyé sur l'hôte distant (ou 0 s'il n'a pas
               été envoyé).
\end{description} ~

Voici par exemple ce que pourra contenir une table :

\noindent \begin{tabular}{| c | c | c |}
     \hline
     e3b0c44298fc1c149afbf4c8996fb92427ae41e4649b934ca495991b7852b855 & try/empty.gpg & 1320932458 \\ \hline
     fe19778cf1ce280658154f2b9c01ffbccd825a23460141dcf3794e7a2c0eb629 & oops.gpg & 1315472654 \\ \hline
     f2ca1bb6c7e907d06dafe4687e579fce76b37e4e93b7605022da52e6ccc26fd2 & test.gpg & 0 \\ \hline
\end{tabular} ~ \newline \newline

On aimerait cependant ne pas avoir à ce soucier du type de la base de données (\textit{MySQL}, \textit{SQLite},
...), il conviendra donc d'utiliser \textit{Django} pour la gestion de la base de données :

\begin{verbatim}
from django.db import models

class DatabaseEntry (models.Model):
     checksum = models.CharField (max_length = 64)
     path     = models.CharField (max_length = 256)
     sent     = models.DateTimeField ()
\end{verbatim}

\section{Intégration Django}

L'interface web étant également développé avec \textit{Django}, il convient donc d'intégrer notre script
dans une application \textit{Django} qui sera distribuée avec l'interface web.

Le projet, \textbf{delikatess}, se présente désormais sous la forme d'un projet \textit{Django} :

\begin{description}
     \item[webui:] L'interface web.
     \item[trackfile:] Le script de sauvegarde et de chiffrement.
     \item[sendit:] Le script d'envoie des fichiers chiffrés.
\end{description}

\subsection{trackfile}

L'application \textit{trackfile} est donc l'intégration de notre script à \textit{Django}.
Ainsi la base de données de notre script est commune à celle de \textit{Django} et de notre
future application web :

\begin{verbatim}
>>> from trackfile.utils import FileManager
>>> fm = FileManager (<gpg-key>, <source directory>, <destination directory>, nbackups = 7)
>>> fm.run ()
\end{verbatim}

\section{Intégration continue}

Afin de vérifier la validité du code source produit à chaque modifications, il va falloir développer
un script qui va devoir remplir les conditions suivantes :

\begin{itemize}
     \item Créer un environnement \textit{Python} virtuel (à l'aide de \textit{virtualenv} dans \textit{.venv}).
     \item Installer les dépendances du projet dans cet environnement (à l'aide du gestionnaire de paquets \textit{Python} : \textit{pip}).
     \item Exécuter les tests unitaires.
\end{itemize}

\begin{verbatim}
#!/bin/sh

echo "===> Creating virtual environment: .venv"
virtualenv .venv || exit 1

echo && echo && echo "===> Installing dependancies"
.venv/bin/pip install -r requirements.txt || exit 1

echo && echo && echo "===> Running test suite"
.venv/bin/python2.7 manage.py test || exit 1
\end{verbatim}

\section{Interface Web}

Maintenant que l'application \textit{trackfile} est terminée (pour le moment), le développement de l'application web
va pouvoir commencer. La création d'une nouvelle application \textit{Django} ne sera pas nécessaire, en effet, il ne
s'agit ici que d'une interface web. La base de donnée étant remplie par l'application \textit{trackfile}, notre interface
ne sera utilisée que pour en visualiser le contenu.

Notre projet \textit{Django} prend donc forme :

\begin{verbatim}
|-+ delikatess
| |-- __init__.py
| |-- manage.py
| |-- settings.py
| |-+ trackfile
| | |-- __init__.py
| | |-- models.py
| | |-- utils.py
| | |-- tests.py
| |-- urls.py
| |-- views.py
\end{verbatim}

Ici, le dossier \textit{trackfile} est celui de l'application, l'interface web ne sera contenue que dans le fichier
\textit{views.py}.

L'application web sera construite à l'aide du front-end \textit{Twitter Bootstrap}, de \textit{dajax} et \textit{dajaxice}
(plugins \textit{AJAX} pour \textit{Django}) :

\begin{description}
     \item[Twitter bootstrap :] \url{http://twitter.github.com/bootstrap}
     \item[dajaxice \& dajax :] \url{http://www.dajaxproject.com}
\end{description}


L'interface web se décomposera en différentes parties (cf. schéma I.2.2) :

\begin{itemize}
     \item Une barre d'outil (informera sur l'utilisateur connecté ainsi que le dossier courrant).
     \item Une vue en liste du contenu de l'espace de stockage.
     \item Une vue en arborescence dont les interactions sont couplées à la vue en liste.
     \item Un volet d'information sur le(s) fichier(s) sélectionné(s).
\end{itemize}

\subsection{La vue en arborescence}

La base de donnée peut être rapportée à une liste de chemin vers les fichiers chiffrés. Histoire
de hiérarchiser cela, à partir de cette liste, on génère un arbre (qui est donc stocké dans un
dictionnaire \textit{Python}, équivalent à du \textit{JSON}).

Ce \textit{JSON} est donc transmis à la template \textit{Django} qui s'occupera (à l'aide de \textit{jQuery})
de générer l'arborescence.

\begin{verbatim}
entries = DatabaseEntry.objects.all ()

# Create JSON tree
json = {}

for dbe in entries:
     folders = dbe.path.split (os.sep)
     d = json

     for name in folders:
          # Is it the end of path ?
          if name == folders[-1]:
               d[name] = (dbe.checksum, dbe.sent)
          # If not in the tree, create it
          elif name not in d:
               d[name] = {}

          d = d[name]
\end{verbatim}
