Maintenant que l'application \textit{trackfile} est terminée (pour le moment), le développement de l'application web
va pouvoir commencer. La création d'une nouvelle application \textit{Django} ne sera pas nécessaire, en effet, il ne
s'agit ici que d'une interface web. La base de donnée étant remplie par l'application \textit{trackfile}, notre interface
ne sera utilisée que pour en visualiser le contenu.

Notre projet \textit{Django} prend donc forme :

\begin{verbatim}
|-+ delikatess
| |-- __init__.py
| |-- manage.py
| |-- settings.py
| |-+ trackfile
| | |-- __init__.py
| | |-- models.py
| | |-- utils.py
| | |-- tests.py
| |-- urls.py
| |-- views.py
\end{verbatim}

Ici, le dossier \textit{trackfile} est celui de l'application, l'interface web ne sera contenue que dans le fichier
\textit{views.py}.

L'application web sera construite à l'aide du front-end \textit{Twitter Bootstrap}, de \textit{dajax} et \textit{dajaxice}
(plugins \textit{AJAX} pour \textit{Django}) :

\begin{description}
     \item[Twitter bootstrap :] \url{http://twitter.github.com/bootstrap}
     \item[dajaxice \& dajax :] \url{http://www.dajaxproject.com}
\end{description}


L'interface web se décomposera en différentes parties (cf. schéma I.2.2) :

\begin{itemize}
     \item Une barre d'outil (informera sur l'utilisateur connecté ainsi que le dossier courrant).
     \item Une vue en liste du contenu de l'espace de stockage.
     \item Une vue en arborescence dont les interactions sont couplées à la vue en liste.
     \item Un volet d'information sur le(s) fichier(s) sélectionné(s).
\end{itemize}

\subsection{La vue en arborescence}

La base de donnée peut être rapportée à une liste de chemin vers les fichiers chiffrés. Histoire
de hiérarchiser cela, à partir de cette liste, on génère un arbre (qui est donc stocké dans un
dictionnaire \textit{Python}, équivalent à du \textit{JSON}).

Ce \textit{JSON} est donc transmis à la template \textit{Django} qui s'occupera (à l'aide de \textit{jQuery})
de générer l'arborescence.

\begin{verbatim}
entries = DatabaseEntry.objects.all ()

# Create JSON tree
json = {"children": []}

for dbe in entries:
     folders = dbe.path.split (os.sep)
     d = json["children"]

     for name in folders:
          tmp = {}

          # Is it the end of the path ?
          if name == folders[-1]:
               tmp["name"] = str (name)
               tmp["hash"] = str (dbe.checksum)
               tmp["sent"] = dbe.sent
               d.append (tmp)
               break

          # Is it already in the tree ?
          alreadyin = False

          for child in d:
               if child["name"] == name:
                    alreadyin = True

                    # If already in, just go to the next level
                    if "children" in child:
                         d = child["children"]

                    break

          # If not already in, create it
          if alreadyin == False:
               tmp["name"] = str (name)
               tmp["children"] = []

               d.append (tmp)
               d = tmp["children"]
\end{verbatim}

Cet arbre \textit{JSON} est ensuite convertit en \textit{HTML} grâce à la librairie \textit{JavaScript} :
\textbf{PURE} (\url{http://beebole.com/pure/}), et avec un peu de \textit{JavaScript} supplémentaire, l'arbre
devient interactif.

\subsection{Le reste de l'interface}

Pour le reste de l'interface, tout se base sur cet arbre. Les deux autres éléments (la table et la barre du haut)
s'intègrent à ce dernier grâce à un peu de \textit{JavaScript} :

\begin{itemize}
     \item Un clic sur un dossier de l'arborescence provoque :

     \begin{itemize}
          \item L'apparition des sous-dossiers dans l'arborescence.
          \item La modification du chemin courant dans la barre du haut.
          \item Le chargement du contenu du dossier dans la table.
     \end{itemize}

     \item Un clic sur un éléments de la barre du haut provoque :

     \begin{itemize}
          \item L'apparition des sous-dossiers dans l'arborescence.
          \item La modification du chemin courant dans la barre du haut.
          \item Le chargement du contenu du dossier dans la table.
     \end{itemize}

     \item L'ouverture d'un dossier dans la table provoque :

     \begin{itemize}
          \item L'apparition des sous-dossiers dans l'arborescence.
          \item La modification du chemin courant dans la barre du haut.
          \item Le chargement du contenu du dossier dans la table.
     \end{itemize}
\end{itemize}

Comme on peut le voir, chaque action sur chaque élément de l'interface fait appel aux mêmes fonctions.


