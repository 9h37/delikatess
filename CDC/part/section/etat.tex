De nombreuses recherches ont été effectuées pour la réalisation du pojet. Certaines solutions ont été retenues, d'autres non.
Résumons en quelques étapes l'avancement du projet :

\begin{enumerate}
     \item Le projet est local à l'entreprise, le stockage des données se fait chez un hôte distant.
     \item Le projet se centralise chez l'hôte distant :
     \begin{itemize}
          \item L'interface web est centralisée, les clients se connectent donc au serveur.
          \item Le client ne possède donc que les données sur lesquelles il travaille.
          \item Le projet devient donc une sorte de gestionnaire de version.
     \end{itemize}

     \item Développement d'une interface web.

     \item Recherche de différents algorithmes pour l'envoi de données efficaces :
     \begin{itemize}
          \item Découpage du fichier en bloc de données.
          \item Comparaison des blocs de données présents chez le client et de ceux présents sur le serveur.
          \item Envoi des blocs qui ont changés.
     \end{itemize}

     \item Réalisation de benchmarks sur l'algorithme retenu, les résultats sont plutôt mauvais.
     \item Analyse de différentes solutions libres :
     \begin{itemize}
          \item \textbf{duplicity} (utilitaire joignant \textit{rdiff-backup} et \textit{GnuPG}) \url{http://duplicity.nongnu.org}
          \item \textbf{bup} (une alternative à \textit{DropBox}, open-source, basée sur \textit{git}) \url{https://github.com/apenwarr/bup}
     \end{itemize}
\end{enumerate}

