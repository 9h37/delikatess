\subsection{Convention de l'ANSSI}

L'Agence Nationale de la Sécurité des Systèmes d'Information recommande l'utilisation de
l'algorithme de hachage \textbf{SHA-256} et l'usage de clé \textbf{RSA 2048 bits}.

\subsection{Consigne}

Le script de base devra être développé en \textit{Python 2.7} et devra remplir les conditions
suivantes :

\begin{enumerate}
     \item Indentation de 4 espaces, pas de tabulations.
     \item Deux répertoires : \textbf{src} et \textbf{dest}, le fichiers ne doivent pas être
     copiés de \textit{src} vers \textit{dest}, ce dernier répertoire ne devra contenir que du
     contenu chiffré.
     \item Calculer la somme SHA-256 des fichiers sources et l'afficher.
     \item Chiffrer les fichiers avec \textit{GnuPG} dans \textit{dest}.

     \begin{enumerate}
          \item Si le fichier n'existe pas, le créer.
          \item Si le fichier existe le nommer selon ce pattern : \textit{filename.X.gpg} où X
          varie de 1 à 5 (à la 6è itération, on supprime le fichier).
     \end{enumerate}

     \item Placer le code source dans un dépôt \textit{git}.
     \item Utiliser \textit{unittest} pour les tests unitaires.
\end{enumerate}

\subsection{Conclusion}

Cette partie du logiciel a été développée sous la forme d'un paquet \textit{Python}.

A l'aide d'un unique objet, on est capable de récupérer la liste des fichiers du répertoire
source, de calculer leurs sommes SHA-256 et de les chiffrer via \textit{GnuPG} :

\begin{verbatim}
from src.filemanager import FileManager

fm = FileManager.FileManager (srcpath, destpath)
filelist = fm.read_entries ()
checksum = fm.hash_entries (filelist)
fm.gpg_encrypt (filelist) # encrypt
\end{verbatim}





