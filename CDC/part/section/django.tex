L'interface web étant également développé avec \textit{Django}, il convient donc d'intégrer notre script
dans une application \textit{Django} qui sera distribuée avec l'interface web.

Le projet, \textbf{delikatess}, se présente désormais sous la forme d'un projet \textit{Django} :

\begin{description}
     \item[webui:] L'interface web.
     \item[trackfile:] Le script de sauvegarde et de chiffrement.
     \item[sendit:] Le script d'envoie des fichiers chiffrés.
\end{description}

\subsection{trackfile}

L'application \textit{trackfile} est donc l'intégration de notre script à \textit{Django}.
Ainsi la base de données de notre script est commune à celle de \textit{Django} et de notre
future application web :

\begin{verbatim}
>>> from trackfile.utils import FileManager
>>> fm = FileManager (<gpg-key>, <source directory>, <destination directory>, nbackups = 7)
>>> fm.run ()
\end{verbatim}
