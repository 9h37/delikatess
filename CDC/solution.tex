\part{Solution proposée}

Dans cette partie, il sera proposé une solution permettant de répondre à la demande
présentée ci-dessus.

\section{Mise en place du serveur}

Le choix du matériel dédié au serveur reste à déterminer (en fonction du volume de données à
traiter). \newline

Le serveur devra tourner sous un \textit{UNIX}/\textit{Linux}. Je proposerai ici
\textit{Debian GNU/Linux}.

La politique de \textit{Debian}, concernant la gestion des paquets, nous assurent que les
logiciels présents dans les dépôts stables de la distribution (\textit{squeeze}) incorporent
de nombreux correctifs de sécurité.

L'installation de cette distribution est rapide (maximum 1h selon mon expérience) et facile
(toutes les tâches de configuration de base sont faites automatiquement : réseau, dépôts, paquets
additionnels, ...).

\section{Partage de données}

Pour le service de partage réseau, on utilisera \textit{Samba}, afin de permettre aux utilisateurs
\textit{Windows} de se connecter au partage (les utilisateurs \textit{UNIX}/\textit{Linux} devront
installer \textit{Samba} sur leurs machines afin de pouvoir monter le partage). \newline

\begin{verbatim}
     # aptitude install samba samba-client
     # mkdir -p /srv/share
     # useradd -g smbusers <username>
     # smbpasswd -a <username>
\end{verbatim}

\paragraph{/etc/samba/smb.conf}
\begin{verbatim}
     [global]
          workgroup = <workgroup>
          netbios name = Partage réseau
          server string = %h server (Samba %v)
          log file = /var/log/samba/log.%m

     [NetDrive]
          path = /srv/share
          browseable = yes
          writeable = yes
          valid users = @smbusers
          invalid users = root
          guest ok = No
\end{verbatim}

\textbf{NB:} Il existe une application web permettant de configurer \textit{Samba} :
SWAT (\textbf{S}amba \textbf{A}dministration \textbf{W}eb \textbf{T}ool).

\section{Sauvegarde et chiffrement des données}

La sauvegarde des données étant périodique, on utilisera \textit{cron} pour planifier
l'exécution de notre script.

Tout d'abord il faut établir le rôle de notre script de sauvegarde.

